\documentclass[xetex,14pt]{beamer}
\usepackage{fontspec}
\usepackage{alltt}
\setbeamertemplate{navigation symbols}{}
\usefonttheme{serif}
\setmainfont{Hoefler Text}
%\renewcommand\ttdefault{lmtt}
\setmonofont[Scale=MatchLowercase%,WordSpace={1,0,0}
]{Menlo}
\newcommand{\const}[1]{\mbox{\fontspec{Abadi MT Condensed Light}#1}}


\title{Ordinals in HOL:
\\transfinite arithmetic up to\\
(and beyond) $\omega_1$}
\author{Michael Norrish \and Brian Huffman}
\date{Wednesday, 24 July 2013}

% 68	66	41
% 44 42 29

\definecolor{DarkBrown}{RGB}{86,83,55}
\definecolor{PaperBrown}{RGB}{226,225,199}
\definecolor{MyPurple}{RGB}{75,30,187}
\setbeamercolor{normal text}{fg=DarkBrown}
\setbeamercolor{structure}{fg=DarkBrown}
\setbeamercolor{background canvas}{bg=PaperBrown}
\setbeamertemplate{frametitle}[default][center]
\usebackgroundtemplate{%
  \includegraphics[width=\paperwidth,height=\paperheight]{keynotepaper}}

\setbeamertemplate{footline}{%
\leavevmode%
\hbox{%
\begin{beamercolorbox}[wd=.50\paperwidth,ht=2.25ex,dp=1ex,right]{section in head/foot}%
\color{gray}ITP~$2014$\end{beamercolorbox}%
\begin{beamercolorbox}[wd=.50\paperwidth,ht=2.25ex,dp=1ex,right]{section in head/foot}%
\color{gray}$\insertframenumber$/$\inserttotalframenumber$\hspace*{1em}%
\end{beamercolorbox}}%
\vskip0pt%
}

\newenvironment{mytt}{\color{MyPurple}\fontsize{12}{12}\selectfont\begin{alltt}}{\end{alltt}}
\newcommand{\ty}[1]{\texttt{\color{MyPurple} #1}}

\begin{document}
\begin{frame}
\maketitle
\end{frame}

\begin{frame}{Why?}
Ordinals are \textbf{cool}: where else can we say something as mind-blowing as \emph{“the set of countable ordinals is uncountable”}\,?

\bigskip
Previous approaches in typed higher order logics have not allowed
\begin{itemize}
\item suitably arbitrary uses of supremum; or
\item modelling of $\omega_1$
\end{itemize}
\end{frame}

\begin{frame}{Also, Ordinals in ACL2}

ACL2 uses ordinals to justify recursive definitions:
\begin{enumerate}
\item find a suitable ordinal when making definition (automatically or interactively);
\item system admits definition
\end{enumerate}

\bigskip But, ACL2’s ordinals are actually an ordinal notation, with no verified connection to “real” ordinals.
\end{frame}

\begin{frame}{ACL2’s Ordinals}

ACL2’s notation is Cantor Normal Form up to $\varepsilon_0$
\begin{itemize}
\item \emph{e.g.}, $\omega^2 + \omega\cdot 2 + 1$ or $\omega^{\omega^{\omega+1}} + \omega^3\cdot 4 + \omega \cdot 10 + 4$
\end{itemize}

\bigskip Kaufmann and Slind show that $<$ on this type is well-founded; this is all that’s really necessary.

\bigskip
However, we \emph{have} shown the ACL2 type and operations are valid ordinal arithmetic.

\end{frame}
\begin{frame}[fragile]
\frametitle{Notational Approaches}
Generally, a notational approach is easy to mechanise.

\bigskip
Do the equivalent of
\begin{mytt}
  Hol_datatype`ord = End of num
                   | Plus of ord \(\times\) num \(\times\) ord`
\end{mytt}

\bigskip
But, this only captures countably many ordinals.
\end{frame}
\newcommand{\num}{\ty{num}}
\newcommand{\bool}{\ty{bool}}
\begin{frame}[fragile]
\frametitle{Another Algebraic Approach}

Based on understanding of ordinals as \emph{‘just like the naturals with a \const{sup} (or \const{limit}\,) function’}.

\begin{mytt}
  Hol_datatype`ord = Z
                   | S of ord
                   | Lim of (num \(\to\) ord)`
\end{mytt}
Using \ty{num} above still only gets countable ordinals (\,and \const{sup} over countable sets).

\bigskip
More importantly, tricky quotienting still required (\,see paper for how to make this work).

\end{frame}

\begin{frame}{von Neumann’s Approach}
An ordinal number is a set $\alpha$ such that
\begin{itemize}
\item $\alpha$ is transitive (that is, every member of $\alpha$ is also a subset of $\alpha$); and
\item
$\forall x,\,y \in \alpha$ one of the following holds: $x \in y$, $x=y$ or $y \in x$.
\end{itemize}

\bigskip
And so, every ordinal is equal to the set of its own predecessors.
\end{frame}

\begin{frame}{Simple Types and von Neumann}

If the type of an ordinal $\alpha$ has to equal the type of a set of ordinals ($\alpha$’s predecessors), we must solve “\ty{$\tau$ set $=\; \tau$}”, which is clearly impossible in HOL.

\bigskip
The best we can hope for is to show that ordinals are in bijection with predecessor sets\dots

\end{frame}
\begin{frame}{von Neumann is a Distraction}

“Really,” ordinals are just canonical wellorders of a given order type.

\bigskip
In set theory (ZFC, NBG, ...) we can’t say “\emph{ordinals are equivalence classes of wellorders}” because this phrase does not denote a set.

\bigskip
But we \emph{can} do just this in HOL.

\end{frame}

\begin{frame}{Ordinals \emph{are} Wellorder Equivalence Classes}

This works in HOL because the wellorders, and thus the ordinals, are with respect to some underlying set.

\bigskip
Start with \ty{$\alpha$ wellorder}, the type of sets of pairs of \ty{$\alpha$}s where the relation is a wellorder.

\bigskip
And so, the \ty{$\alpha$ wellorder}s are in bijection with a (strict) subset of all possible values of type \ty{$(\alpha \times \alpha)$~set}.
\end{frame}

\begin{frame}{Necessary Properties of Wellorders}
Need to \textbf{define} notions of
\begin{itemize}
\item wellorder isomorphism;
\item initial segments on wellorders; and
\item wellorder $<$:  $u < v$ iff there is an $e$ in $v$ such that $u$ is order isomorphic to the initial segment of $v$ up to $e$
\end{itemize}

\bigskip
Need to \textbf{prove}:
\begin{itemize}
\item isomorphism an equivalence;
\item ordering is a partial order, well-founded, trichotomous.
\end{itemize}
\end{frame}

\begin{frame}{Next Step: Quotient}
All the important properties lift through quotienting.

\bigskip
Thanks to well-foundedness, can define \const{oleast} operator, returning minimal ordinal of a non-empty set.
\begin{itemize}
\item $\const{oleast}\{x\mid\const{T}\}$ is the zero ordinal.
\end{itemize}
\end{frame}

\newcommand{\ainf}{\ty{$\alpha\;\,+$~num}}
\begin{frame}{Cardinalities}
If the type \ty{$\alpha$} is finite, \ty{$\alpha$~wellorder} only has finitely many elements too.

\bigskip
So, let the \ty{$\alpha$~ordinal} type be a quotient of wellorders over the (sure to be infinite) type \ainf.
\begin{itemize}
\item $\const{oleast}\{ x \mid y < x \}$ is the successor of $y$
\item some work (still to come) to show this always exists
\end{itemize}

\end{frame}

\begin{frame}{The Critical Cardinality Result}
There are strictly more values in \ty{$\alpha$~ordinal} than there are in \ainf
\begin{itemize}
\item follows from the observation that \ty{$\alpha$~ordinal} itself forms a wellorder, and
\item
that every wellorder over \ainf{} is isomorphic to an initial segment of the \ty{$\alpha$~ordinal} wellorder
\end{itemize}

\end{frame}

\begin{frame}{Defining Supremum}

Let
\[
\const{sup}\;S = \const{oleast}\{ \alpha \mid \alpha \not\in \bigcup_{\beta\in S}\const{preds}\;\beta\}
\]

\bigskip
\emph{I.e.}, the least ordinal not in the combined predecessors of all the elements in $S$.

\end{frame}

\begin{frame}{Supremum Works}

“\emph{The least ordinal not in the combined predecessors of all the elements in $S$}” is OK because:
\begin{itemize}
\item any given ordinal in \ty{$\alpha$~ordinal} has no more predecessors than \ainf; and
\item
cardinal $\kappa\times\kappa\approx\kappa$, so there must be a minimal element not in the collective predecessors
\end{itemize}

\end{frame}

\begin{frame}{The Supremum Rule}

It is legitimate to write
\[
\const{sup}\; S
\]
when $S$ is a set of \ty{$\alpha$~ordinal}s if
\[
S \preceq \ainf
\]

\end{frame}

\begin{frame}{And so\dots}
Can define $\omega = \const{sup} \{\const{\&}n \mid \const{T} \}$
\begin{itemize}
\item where \const{\&} is the injection from natural numbers into ordinals
\end{itemize}

\bigskip
Can distinguish limit and successor ordinals.

\bigskip
Can prove a recursion theorem by cases\dots

\end{frame}

\begin{frame}{A Recursion Theorem}

With $<$ on ordinals well-founded, one could always define functions by well-founded recursion.

\bigskip
\onslide<2->
However, this pseudo-algebraic principle is nicer to use:
\[
\begin{array}{l}
\forall z\,\mathit{sf}\,\mathit{lf}.\;\exists! f.\\
\quad\begin{array}[t]{rcl}
f(0) &=& z\\
f(\alpha^+) &=& \mathit{sf}(\alpha,f(\alpha))\\
f(\beta) &=& \mathit{lf}(\beta, \{\, f(\eta)\mid \eta < \beta\})
\end{array}
\end{array}
\]
(where $\beta$ has to be a non-zero limit ordinal).

\end{frame}

\begin{frame}{Arithmetic Comes Next}
The recursion principle makes it easy to define
\begin{itemize}
\item addition,
\item multiplication,
\item exponentiation
\end{itemize}

\bigskip
Some more work results in definitions and properties of division, remainder, and discrete logarithm.

\end{frame}

\begin{frame}{See Paper For:}
\textbf{Cantor Normal Forms}:
\begin{itemize}
\item Every ordinal can be expressed as a unique “polynomial” over bases $\geq 2$
\end{itemize}

\bigskip\bigskip\onslide<2->
\textbf{Existence of Fixed Points}:
\begin{itemize}
\item Every increasing, continuous function has infinitely many fixed points
\item \emph{E.g.}, can define $\varepsilon_0$, first fixed point for $x \mapsto \omega^x$
\end{itemize}

\end{frame}

\begin{frame}{Countable Ordinals and $\omega_1$}

A \emph{countable ordinal} is one with countably many predecessors.

\bigskip
In \ty{$\alpha$~ordinal}, which is over \ainf, all ordinals may be countable.
\begin{itemize}
\item Critical cardinality result tells us there are uncountably many of them!
\end{itemize}

\bigskip
To get more, instantiate \ty{$\alpha$} in \ainf{} to \ty{$\alpha + (\texttt{num}\to\texttt{bool})$}

\end{frame}

\begin{frame}{The First Uncountable Ordinal}


First, prove that cardinality of $\{\beta \mid \beta\mbox{ is countable}\}$ is $\preceq$
cardinality of \ty{$(\alpha + (\texttt{num} \to \texttt{bool})) + \texttt{num}$}

\bigskip
Then, it’s legitimate to write
\[
\omega_1 \stackrel{\mbox{\tiny def}}{=} \const{sup}\{\beta \mid \beta \mbox{ is countable}\}
\]
when $\beta$ has type \ty{$(\alpha + (\texttt{num}\to\texttt{bool}))\;\texttt{ordinal}$}

\end{frame}

\begin{frame}{$\omega_1$ and so on}

$\omega_1$ is the first uncountable ordinal:
\[
\beta < \omega_1 \iff \beta \mbox{ is countable}
\]

\bigskip
To capture $\omega_2$ we might instantiate type variable
\[
\ty{$\alpha$} \mapsto \ty{$\alpha + ((\num\to\bool)\to\bool)$}
\]

\end{frame}

\begin{frame}{Conclusions}

The “obvious” way to mechanise ordinals, as equivalence classes of wellorders, works well.

\bigskip
Supremum can be defined naturally, taking sets of ordinals as an argument.
\begin{itemize}
\item Usual arithmetic falls out
\end{itemize}

\bigskip
Just as naturally, large ordinals such as $\omega_1$ can be defined.

\end{frame}

\end{document}
%%% Local Variables:
%%% coding: utf-8
%%% mode: latex
%%% TeX-engine: xetex
%%% End:
